\documentclass{ltjsarticle}
% ltjsarticle: lualatex 用の 日本語 documentclass
% 他のタイプセットエンジンを使ってビルドする場合は、 \documentclass[dvipdfmx]{jsarticle} などとする。
\usepackage{mathrsfs}

\begin{document}

\title{Chapter 10 The Blahut-Arimoto Alogorithmes}
\author{Tetsuya SHIMIZU}
\maketitle

離散的無記憶通信路$p(y|x)$の場合,通信路容量は
\begin{eqnarray}
  tab   
  C = \max_{r(x)} I(X;Y)
\end{eqnarray}
となる.ここで$X$と$Y$はそれぞれ一般的な通信路の入力と出力,$r(x)$は入力分布であり,通信路を通じて情報を確実に伝送できる漸近的に達成可能な最大レートを特徴付ける.



\section{Introduction}
量子情報検定問題を解くことで,光デジタル通信システムの可能性を推定し,電磁場の量子力学的性質に起因する性能の根本的な限界を定式化することができるため,それ自体が興味深い.
しかし,最近まで,厳密な解が得られているケースはほとんどありませんでした.

観測データ$y$の空間上のいくつかの確率分布$P_1(dy), \dots, P_M(dy)$の中から選択する問題の量子的一般化,あるいは多重統計的仮説検定の問題がHelstromによって定式化された.
$M$個の仮説が,観測可能な量子系(例えば,開口部の量子電磁場)の1個の可能な統計的状態に関連しているとする.
一般に,量子純粋状態は,観測空間と呼ばれる複素ヒルベルト空間$\mathscr{H}$に作用する密度演算子$\rho$によって記述される.
この密度演算子は,非負定数$\rho \geq 0$であり,単位トレースを持つことを忘れてはならない.
$\mathrm{Tr} \rho = 1$であり,非量子(古典的な場合)の確率分布に相当する.

量子状態$\rho_1, \dots, \rho_M$に関連する仮説検定問題の数学的解決は,古典的な統計的決定関数に類似したエルミート「決定」演算子$\Pi_1, \dots,\Pi_M$を見つけることに帰着する.
演算子$\{ \Pi_j, j = 1, M \}$は,決定条件付き確率$\mathrm{Pr} \{j | k \}$を定義できる.
\begin{equation}
  \mathrm{Pr} \{ j | k\} = \mathrm{Tr} \Pi_j \rho_k = p_{jk}  
\end{equation}
ここで,次の条件を満たす.$\Pi_j \geq 0 \mathrm{for all} j = 1, M$と$\sum_{j=1}^{M} \Pi_j = 1$ここで1は恒等演算子である.
現代的なアプローチとは異なり,直行性$\Pi_k \Pi_jl = 0 \mathrm{for all} k \neq l$と可換性$\Pi_k \Pi_l = \Pi_l \Pi_k$の条件は必要ない.
これは,アクセス可能案別のシステムに対する観測を犠牲にして,量子間接測定,または純測定の日付に基づいて,ランダム化された決定規則を認めることに相当する.
このような量子測定と決定則の拡張により,受信した量子信号の非可換観測量を推定することができる.

一般的なベイスアプローチのよれば,決定演算子$\{\Pi_j\}$の集合の中から最適な集合$\{ \Pi_j^0\}$を選択するためには,コスト$\{ C_{jk} \}$と事前確率$\{ \pi_k \}$の行列を与え,平均コスト(またはリスク)を最小化することが必要である.

\begin{equation}
  C = \sum_{j,k=1}^{M} P_{jk} C_{jk} \pi_k = \mathrm{Tr} \sum_{j,k=1}^{M} C_{jk} \rho_k \pi_k
\end{equation}

$M=2$の場合,この最小化問題の解はHelstromによって発見されている.また,純粋な量子状態の2値検出は[8]で検討されている.
M-Alternativeの場合($M>2$)には,全ての$k,l$に対して$\rho_k \rho_l = \rho_l \rho_k$という特殊な可換ケースを除いて,最近まで正確な解はなかった.
この可換ケースは実用的であるが,量子論の観点からは退化的である.
可換だるため,全ての密度演算子$\{\rho_k\}$は共通の対角表現を持ち,その中で非量子確率分布$\{P_k(dy)\}$で表され,その最適判別は古典統計決定理論の通常のルールに従って$y$を観測することによって実行されるからである.

この論文には,より一般的で厳密な結果が含まれている.ここでは,多数の代替的な量子統計的仮説検定の最適化に関するいくつかの一般的なケースを扱い,それらは正確に解決される.
次に,必要な概念を導入し,多重検定の最適化の条件を定式化し,分析する.
次に,一般的な量子最適化問題を,2次元の決定空間の場合について解く.
最後の段階は,光学的確率信号のM-ary検出の一般的な問題である.
この問題の解は,準古典近似(強力な信号)と極限量子限界(弱い信号)の2つの漸近的なケースで見いだされる.

\section{The conditions of optimality and sufficient decision spaces}
決定演算子 $\Pi_j^0$の集合が多重量子仮説検定問題の最適解となるための必要十分条件は,最近[2,3,10,11]において報告されたものである.
これらの条件を以下の式で書く.
\begin{eqnarray}
  (A_j - \Lambda) \Pi_j = 0, \ \ \Pi_j(A_j - \Lambda) = 0 \\
  A_j - \Lambda \geq 0\ \ \ \forall  j = 1, M
\end{eqnarray}
ここで,$A_j = \sum_{k=1}^{M} C_{jk} \rho_k \pi_k $を事後リスクのエルミート演算子とする.
式(3)を$j$に拡張し,$\sum_{j=1}^{M} \Pi_j = 1$という条件での和で定義されるため,演算子$\Lambda$はエルミート型であることがわかる.
\begin{eqnarray}
  \Lambda = \sum_{j=1}^{M} A_j \Pi_j^0 = \sum_{j=1}^{M} \Pi_j^0 A_j
\end{eqnarray}
ここで,$\{ \Pi_j^0\}$はこれらの方程式の解である.

式(3)(4)の充足性はほぼ明らかである.
演算子$\{\Pi_j^0\}$が式(3)を満たす場合,対応する平均コスト(2)は$\mathrm{Tr}\Lambda$となり,任意の集合$\{\Pi_j\}$について,少なくとも一つの演算子$\{\Pi_j\}$について,$(A_j - \Lambda) \Pi_j \neq 0$であるとき,コストの差$C = \mathrm{Tr}\sum_{j}A_j \Pi_j$,$C^0$は以下の式を満たす.
\begin{eqnarray}
  C - C^0 = \mathrm{Tr}
  \left(
    \sum_{j = 1}^{M} A_j \Pi_j - \Lambda
  \right) =
  \sum_{j=1}^{M} \mathrm{Tr} (A_j - \Lambda) \Pi_jk
\end{eqnarray}
ここで,$A_j - \Lambda \geq 0, \Pi_j \geq 0$ならば正である.

\end{document}