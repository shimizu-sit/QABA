\documentclass{ltjsarticle}
% ltjsarticle: lualatex 用の 日本語 documentclass
% 他のタイプセットエンジンを使ってビルドする場合は、 \documentclass[dvipdfmx]{jsarticle} などとする。
\usepackage{graphicx}
\usepackage{mathrsfs}
\usepackage{amssymb}

\usepackage{amsthm}

\theoremstyle{definition}
\newtheorem{dfn}{Definition}[section]
\newtheorem{prop}[dfn]{Proposition}
\newtheorem{lem}[dfn]{Lemma}
\newtheorem{thm}[dfn]{Theorem}
\newtheorem{cor}[dfn]{Corollary}
\newtheorem{rem}[dfn]{Remark}
%\newtheorem*{rem*}[dfn]{Remark}
\newtheorem{fact}[dfn]{Fact}
\renewcommand{\qedsymbol}{$\blacksquare$}

\begin{document}

\title{Chapter 10 The Blahut-Arimoto Alogorithmes}
\author{Tetsuya SHIMIZU}
\maketitle

離散的無記憶通信路$p(y|x)$の場合,通信路容量は
\begin{eqnarray}
  \label{10.1}
  C = \max_{r(x)} I(X;Y)
\end{eqnarray}
となる.
ここで,$X$と$Y$はそれぞれ一般的な通信路の入力と出力,$r(x)$は入力分布であり,通信路をとおして情報を確実に伝送できる漸近的に達成可能な最大レートを特徴づける.
式(\ref{10.1})の$C$は,汎用通信路の遷移行列のみに依存し,通信路に対する符号のブロック長$n$には依存しないので,\textit{single-letter characterization}(一文字特性化)と呼ばれる.
入力アルファベット$\mathcal{X}$と出力アルファベット$\mathcal{Y}$の両方が有限であるとき,$C$の計算は有限次元の最大化問題となる.

一般的な確率変数$X$を持つi.i.d.情報源$\{X_k, k \geq 1\}$に対して,レート歪み関数
\begin{eqnarray}
  \label{10.2}
  R(D) = \min_{Q(\hat{x}|x): Ed(X,\hat{X}) \leq D} I(X;\hat{X})
\end{eqnarray}
1文字の歪み尺度$d$に関して$D$以下の平均歪みで情報源を再生するレート歪み符号の漸近的に達成できる最小レートを特徴付ける.
再び,式(\ref{10.2})の$R(D)$が一文字特性であり,これはレート歪み符号のブロック長$n$でゃなく,一般的な確率変数$X$だけに依存する.
情報源アルファベット$\mathcal{X}$と再生アルファベット$\hat{\mathcal{X}}$の両方が有限であるとき,$R(D)$の計算は有限次元の最小化問題となる.

よほど特殊な場合を除き,$C$や$R(D)$の式を閉じた形で得ることは不可能であり,数値計算に頼らざるを得ない.
しかし,これらの量を計算することは,関連する最適化問題が非線形であるため,一筋縄ではいかない.
本章では,このために考案された反復アルゴリズムであるBlahut-Arimotoアルゴリズム(以下,BAA)について解説する.

BAAがどのように,そしてなぜ機能するのかをよりよく理解するために,まず次のセクションで一般的な設定におけるアルゴリズムを説明する.
$C$と$R(D)$の計算のためのアルゴリズムの特殊化については2節で説明し,アルゴリズムの収束については3節で証明する.

\section{Alternating Optimization}
本節では,交互最適化アルゴリズムについて説明する.
このアルゴリズムは,次節で通信路容量とレート歪み関数を計算するために特化される予定である.
二重の$\sup$を考える.
\begin{eqnarray}
  \label{10.3}
  \sup_{\mathbf{u}_1 \in A_1} \sup_{\mathbf{u}_2 \in A_2} f(\mathbf{u}_1, \mathbf{u}_2)
\end{eqnarray}
ここで,$A_i$は$i = 1,2$について$\mathfrak{R}^{n_i}$の凸部分集合であり,$f$は$A_1 \times A_2$上で定義された関数である.
関数$f$は上方から有界であり,$A_1 \times A_2$上で連続的な偏導関数を持っている.
さらに,すべての$\mathbf{u}_2 \in A_2$に対して,
\begin{eqnarray}
  \label{10.4}
  f(c_1(\mathbf{u}_2), \mathbf{u}_2) = \max_{\mathbf{u}_1^{\prime} \in A_1} f(\mathbf{u}_1^{\prime}, \mathbf{u}_2)
\end{eqnarray}
を満たすような一意の$c_1 (\mathbf{u}_2) \in A_1$が存在し,すべての$\mathbf{u}_1 \in A_1$に対して,
\begin{eqnarray}
  \label{10.5}
  f(\mathbf{u}_1, c_2(\mathbf{u}_1)) = \max_{\mathbf{u}_2^{\prime} \in A_2} f(\mathbf{u}_1, \mathbf{u}_2^{\prime})
\end{eqnarray}
を満たすような一意の$c_2(\mathbf{u}_1) \in A_2$が存在するとする.

$\mathbf{u} = (\mathbf{u}_1, \mathbf{u}_2)$と$A = A_1 \times A_2$とすると,式(\ref{10.3})は式(\ref{10.6})と書くことができる.
\begin{eqnarray}
  \label{10.6}
  \sup_{\mathbf{u} \in A} f(\mathbf{u})
\end{eqnarray}

すなわち,$f$の$\sup$は,$\mathfrak{R}^{n_1}$と$\mathfrak{R}^{n_2}$の2つの凸部分集合のデカルト積に等しい$\mathfrak{R}^{n_1 + n_2}$の部分集合に対してそれぞれ取られる.

ここで,式(\ref{10.3})の二重$\sup$の値である$f^{*}$を計算するための交互最適化アルゴリズムについて説明する.
$k \geq 0$に対して$\mathbf{u}^{(k)} = ( \mathbf{u}_1^{(k)}, \mathbf{u}_2^{(k)} )$と定義する.
$A_1$において任意に選んだベクトルを$\mathbf{u}_1^{(0)}$と$\mathbf{u}_2^{(0)}=c_2 (\mathbf{u}_1^{(0)} )$する.
$k \geq 1$の場合,$\mathbf{u}^{(k)}$は
\begin{eqnarray}
  \label{10.7}
  \mathbf{u}_1^{(k)} = c_1 ( \mathbf{u}_2^{(k-1)} )
\end{eqnarray}
と
\begin{eqnarray}
  \label{10.8}
  \mathbf{u}_2^{(k)} = c_2 ( \mathbf{u}_1^{(k)} )
\end{eqnarray}
で定義される.
すなわち,$\mathbf{u}_1^{(k)}$,$\mathbf{u}_2^{(k)}$は,$A_1$において$\mathbf{u}_1^{(0)}$が任意に選ばれることを除いて,順序$\mathbf{u}_1^{(0)}, \mathbf{u}_2^{(0)}, \mathbf{u}_1^{(1)}, \mathbf{u}_2^{(1)}, \mathbf{u}_1^{(2)}, \mathbf{u}_2^{(2)}, \cdots$で生成され,順序の各ベクトルは前のベクトルの関数である.
\begin{eqnarray}
  \label{10.9}
  f^{(k)} = f( \mathbf{u}^{(k)})
\end{eqnarray}
とする.次に式(\ref{10.4})と式(\ref{10.5})から$k \geq 1$について
\begin{eqnarray}
  \label{10.10}
  f^{(k)} &=& f\left(\mathbf{u}_1^{(k)}, \mathbf{u}_2^{(k)}\right) \\
  \label{10.11}
  &\geq& f \left(\mathbf{u}_1^{(k)}, \mathbf{u}_2^{(k-1)}\right) \\
  \label{10.12} 
  &\geq& f \left(\mathbf{u}_1^{(k-1)}, \mathbf{u}_2^{(k-1)}\right) \\
  \label{10.13}
  &=& f^{(k-1)}
\end{eqnarray}
とする.
数列$f^{(k)}$は非減少であり,$f$は上から有界であるから収束するはずである.
3節で$f$が凹型であれば$f^{(k)} \rightarrow f^*$であることを示すことにする.
図\ref{fig10.1}は交互最大化アルゴリズムの説明図であり,この場合$n_1$も$n_2$も$1$に等しく,$f^{(k)} \rightarrow f^*$となる.
\begin{figure}[htbp]
  \label{fig10.1}
  \begin{center}
  \includegraphics{Fig10-1.pdf}
  \caption{Alternating Optimization}
  \end{center}
\end{figure}

交互最適化アルゴリズムは,次のような系図で説明する事ができる.
あるハイカーが山頂を目指すとする.
山のある地点から出発して,南北と東西に交互に移動する.
(この問題では,南北と東西の方向は多次元的である.)
それぞれの移動で,ハイカーは可能な限り高い地点に移動する.
問題は,ハイカーが最終的に山のどの地点からでも山頂に近づくことができるかどうかである.

式(\ref{10.3})の$f$を$-f$に置き換えると,二重$\sup$は二重$\inf$を用いて
\begin{eqnarray}
  \label{10.14}
  \inf_{\mathbf{u}_1 \in A_1} \inf_{\mathbf{u}_2 \in A_2} f \left( \mathbf{u}_1, \mathbf{u}_2 \right)
\end{eqnarray}
となる.$A_1$,$A_2$,$f$に関するこれまでの仮定は,$f$が上方から有界でなく下方から有界であることを仮定した以外はすべて有効である.
式(\ref{10.14})の二重$\inf$は,同じ交互最適化アルゴリズムで計算することができる.
なお,$f$を$-f$に置き換えると,式(\ref{10.4})と式(\ref{10.5})の最大値は最小値になり,式(\ref{10.11})と式(\ref{10.12})の不等号は逆になる.

\section{THE ALGORITHMS}
本節では,前節で説明した交互最適化アルゴリズムを通信路容量とレート歪み関数を計算するために特殊化する.
対応するアルゴリズムはBAアルゴリズムとして知られている.

\subsection{CHANNEL CAPACITY}
以下,入力分布$r(x)$を$\mathbf{r}$で表し,$\mathbf{r}$が厳密に正であれば$\mathbf{r} > 0$,すなわちすべての$x \in \mathcal{X}$に対して$r(x) > 0$と書くことができる.
$\mathbf{r}$が厳密に性でない場合は$\mathbf{r} \geq 0$と書くことにする.

\begin{lem}
  \label{Lemma10.1}
  $r(x)p(x|y)$を$\mathbf{r} > 0$となるよう$\mathcal{X} \times \mathcal{Y}$上に与えられた合同分布とし,$\mathbf{q}$を$\mathcal{Y}$から$\mathcal{X}$への遷移行列とする.
  すると
  \begin{eqnarray}
    \label{10.15}
    \max_{\mathbf{q}} \sum_{x} \sum_{y} r(x) p(x|y) \log \frac{q(x|y)}{r(x)} = \sum_{x} \sum_{y} r(x) p(x|y) \log \frac{q^*(x|y)}{r(x)}
  \end{eqnarray}
  は
  \begin{eqnarray}
    \label{10.16}
    q(x|y) = 0 \mbox{if and only if} p(x|y) = 0
  \end{eqnarray}
  と
  \begin{eqnarray}
    \label{10.17}
    q^*(x|y) = \frac{r(x)p(y|x)}{\sum_{x'} r(x')p(y|x')}
  \end{eqnarray}
  のようなすべての$\mathbf{q}$にわたって最大化がとられ,すなわち最大化する$\mathbf{q}$は入力分布$\mathbf{r}$と遷移行列$p(x|y)$に対応するものであることがわかる.
\end{lem}

式(\ref{10.15})移行では,$r(x) > 0$かつ$p(y|x) > 0$であるすべての$x$と$y$に対して総和をとるという慣例を採用sるう.
なお,式(\ref{10.15})の右辺は,$\mathbf{r}$が一般通信路に対する入力分布である$p(y|x)$とするときの相互情報量$I(X;Y)$を示す.

\begin{proof}
  式(\ref{10.17})を以下のようにする.
  \begin{eqnarray}
    \label{10.18}
    w(y) = \sum_{x'}r(x')p(y|x')
  \end{eqnarray}
  全て$y \in \mathcal{Y}$について一般性を損なって,いくつかの$x \in \mathcal{X}$について$p(y|x) > 0$であると仮定する.
  $\mathbf{r} > 0$なので,すべての$y$について$w(y) > 0$となり,したがって$q^*(x|y)$は定義される.
  式(\ref{10.17})を並べ替えと
  \begin{eqnarray}
    \label{10.19}
    r(x)p(y|x) = w(y)q^*(x|y)
  \end{eqnarray}
  が得られる.

  ここで,
  \begin{eqnarray}
    && \sum_{x} \sum_{y} r(x)p(y|x) \log \frac{q^*(x|y)}{r(x)} - \sum_{x} \sum_{y} r(x)p(y|x) \log \frac{q(x|y)}{r(x)} \nonumber \\
    \label{10.20}
    &=& \sum_{x} \sum_{y} r(x) p(y|x)\log \frac{q^*(x|y)}{q(x|y)} \\
    \label{10.21}
    &=& \sum_{y} \sum_{x} w(y) p^*(x|y)\log \frac{q^*(x|y)}{q(x|y)} \\
    \label{10.22}
    &=& \sum_{y} w(y) \sum_{x}  p^*(x|y)\log \frac{q^*(x|y)}{q(x|y)} \\
    \label{10.23}
    &=& \sum_{y} w(y) D \left(  q^*(x|y) \parallel q(x|y) \right) \\
    \label{10.24}
    &\geq& 0
  \end{eqnarray}
  を考える.式(\ref{10.21})は式(\ref{10.19})から導かれ,最後のステップは発散不等式の適用である.
  そして,式(\ref{10.17})で$\mathbf{r}>0$なので$\mathbf{q}^*$が式(\ref{10.16})を満たすことに注目し,証明を完了する.
\end{proof}

\begin{thm}
  \label{theorem10.2}
  離散的無記憶通信路$p(y|x)$に対して,
  \begin{eqnarray}
    \label{10.25}
    C = \sup_{\mathbf{r}>0} \max_{\mathbf{q}} \sum_{x} \sum_{y} r(x) p(y|x) \log \frac{q(x|y)}{r(x)}
  \end{eqnarray}
  ここで最大化は式(\ref{10.16})を満たすすべての$\mathbf{q}$に対して行われる.
\end{thm}

\begin{proof}
  汎用通信路$p(y|x)$の入力分布$\mathbf{r}$としたときの相互情報量$I(X;Y)$を$I(\mathbf{r}, \mathbf{p})$とする.
  すると,
  \begin{eqnarray}
    \label{10.26}
    C = \max_{\mathbf{r} \geq 0} I (\mathbf{r}, \mathbf{p})
  \end{eqnarray}
  と書くことができる.$\mathbf{r}^*$が$C$を達成するとる.ここで$\mathbf{r}^* > 0$ならば,
  \begin{eqnarray}
    \label{10.27}
    C &=& \max_{\mathbf{r} \geq 0} I(\mathbf{r}, \mathbf{p}) \\
    \label{10.28}
    &=& \max_{\mathbf{r} > 0} I(\mathbf{r}, \mathbf{p}) \\
    \label{10.29}
    &=& \max_{\mathbf{r} > 0} \max_{\mathbf{q}} \sum_{x} \sum_{y} r(x) p(y|x) \log \frac{q(x|y)}{r(x)} \\
    \label{10.30}
    &=& \sup_{\mathbf{r} > 0} \max_{\mathbf{q}} \sum_{x} \sum_{y} r(x) p(y|x) \log \frac{q(x|y)}{r(x)}
  \end{eqnarray}
  となり,式(\ref{10.29})はLemma\ref{Lemma10.1}から成り立つ(そして最大化は式(\ref{10.16})を満たすすべての$\mathbf{q}$に対して行われる).

  次に,$\mathbf{r}^* \geq 0$の場合を考える.
  $((\mathbf{r}, \mathbf{p}))$は$\mathbf{r}$に連続なので,任意の$\epsilon > 0$に対して
  \begin{eqnarray}
    \label{10.31}
    \parallel \mathbf{r} - \mathbf{r}^* \parallel < \delta,
  \end{eqnarray}
  ならば,
  \begin{eqnarray}
    \label{10.32}
    C - I(\mathbf{r}, \mathbf{p}) < \epsilon
  \end{eqnarray}
  となる$\delta > 0$が存在し,ここで$\parallel \mathbf{r} - \mathbf{r}^* \parallel$は$\mathbf{r}$と$\mathbf{r}^*$のユークリッド距離を表す.
  特に式(\ref{10.31})と式(\ref{10.32})を満たす$\hat{\mathbf{r}} > 0$が存在する.
  すると
  \begin{eqnarray}
    \label{10.33}
    C &=& \max_{\mathbf{r} \geq 0} I(\mathbf{r}, \mathbf{p}) \\
    \label{10.34}
    &\geq& \sup_{\mathbf{r} > 0} I(\mathbf{r}, \mathbf{p}) \\
    \label{10.35}
    &\geq& I(\hat{\mathbf{r}}, \mathbf{p}) \\
    \label{10.36}
    &>& C - \epsilon
  \end{eqnarray}
  となり,最後のステップが$\hat{\mathbf{r}}$が式(\ref{10.32})を満たすので成り立つ.
  したがって
  \begin{eqnarray}
    \label{10.37}
    C - \epsilon < \sup_{\mathbf{r} > 0} I(\mathbf{r}, \mathbf{p}) \leq C
  \end{eqnarray}
  が得られる.
  最後に,$\epsilon \rightarrow 0$とすることで,
  \begin{eqnarray}
    \label{10.38}
    C = \sup_{\mathbf{r} > 0} I(\mathbf{r}, \mathbf{p}) = \sup_{\mathbf{r} > 0} \max_{\mathbf{q}} \sum_{x} \sum_{y} r(x) p(y|x) \log \frac{q(x|y)}{r(x)}
  \end{eqnarray}
  となる.
  これで証明は達成される.
\end{proof}
さて,式(\ref{10.3})の二重$\sup$について,
\begin{eqnarray}
  \label{10.39}
  f(\mathbf{r}, \mathbf{q}) = \sum_{x} \sum_{y} r(x) p(y|x) \log \frac{q(x|y)}{r(x)}
\end{eqnarray}
で,$\mathbf{r}$と$\mathbf{q}$がそれぞれ$\mathbf{u}_1$,$\mathbf{u}_2$の役を演じているとする.
\begin{eqnarray}
  \label{10.40}
  A_1 = \{ ( r(x), x \in \mathcal{X} ) : r(x) > 0 \ \ \mbox{and} \ \  \sum_{x} r(x) = 1 \}
\end{eqnarray}
と
\begin{eqnarray}
  \label{10.41}
  A_2 &=& \{ \left( q(x|y), (x,y) \in \mathcal{X} \times \mathcal{Y} \right) : q(x|y) > 0 \nonumber \\
  && \mbox{if} \ \  p(x|y) > 0, q(x|y) = 0 \ \ \mbox{if} \ \ p(y|x) = 0 \nonumber \\
  && \mbox{and} \ \ \sum_{x} q(x|y) = 1 \ \ \mbox{for all} \ \ y \in \mathcal{Y} \}
\end{eqnarray}
そして,$A_1$は$\mathfrak{R}^{| \mathcal{X} |}$の部分集合,$A_2$は$\mathfrak{R}^{|\mathcal{X}| |\mathcal{Y}|}$の部分集合であり,$A_1$,$A_2$ともに凸であることが容易に確認できる.
すべての$\mathbf{r} \in A_1$と$\mathbf{q} \in A_2$について,Lemma\ref{Lemma10.1}により
\begin{eqnarray}
  \label{10.42}
  f(\mathbf{r}, \mathbf{q}) &=& \sum_{x} \sum_{y} r(x) p(y|x) \log \frac{q(x|y)}{r(x)} \\
  \label{10.43}
  &\leq& \sum_{x} \sum_{y} r(x) p(y|x) \log \frac{q^*(x|y)}{r(x)} \\
  \label{10.44}
  &=& I(X;Y) \\
  \label{10.45}
  &\leq& H(X) \\
  \label{10.46}
  &\leq& \log | \mathcal{X} |
\end{eqnarray}
となる.
したがって,$f$は上方から有界である.
すべての$\mathbf{q} \in A_2$について,$p(x|y)=0$となるようなすべての$x$と$y$について$q(x|y)=0$となるので,$\mathbf{q}$のこれらの成分は退化する.
実際,これらの$\mathbf{q}$の成分は式(\ref{10.39})の$f(\mathbf{r}, \mathbf{q})$の定義に現れないが,これは次のように理解できるだろう.
式(\ref{10.39})の二重和は$r(x)>0$かつ$p(y|x)>0$だるようなすべての$x$と$y$に対して行われるという慣例を思い出してほしい.
$q(x|y)=0$なら,$p(y|x)=0$あんおで,該当項は二重和に含まれない.
したがって,式(\ref{10.39})の二重和に含まれる確率はすべて厳密に正であるから,$f$は連続であり,$A$上で連続偏導関数を持つことが容易にわかる.
さらに,任意の$\mathbf{r} \in A_1$に対して,Lemma\ref{Lemma10.1}により,$f$を最大化する一意な$\mathbf{q} \in A_2$が存在することがまもなく示される.

式(\ref{10.3})の二重$\sup$は
\begin{eqnarray}
  \label{10.47}
  \sup_{\mathbf{r} \in A_1} \sup_{\mathbf{q} \in A_2} \sum_{x} \sum_{y} r(x) p(y|x) \log \frac{q(x|y)}{r(x)}
\end{eqnarray}
となり,定理\ref{theorem10.2}により$C$に等しく,全ての$\sup$が実際に最大となる.
まず,$A_1$において厳密に正の入力分布を任意に選び,それを$\mathbf{r}^{(0)}$とする.
次に,Lemma\ref{Lemma10.1}に基づき,$\mathbf{q}^{(0)}$,一般には$k \geq 0$に対して$q^{(k)}$により
\begin{eqnarray}
  \label{10.48}
  q^{(k)}(x|y) \ \frac{r^{(k)}(x) p(y|x)}{\sum_{x'} r^{(k)}(x') p(y|x')}
\end{eqnarray}
を定義する.
$k \geq 1$に対して$\mathbf{r}^{(1)}$,一般的には$\mathbf{r}^{(k)}$を定義するためには,与えられた$\mathbf{q} \in A_2$に対して$f$を最大化する$\mathbf{r} \in A_1$を求める必要があり,$\mathbf{r}$に対する制約は
\begin{eqnarray}
  \label{10.49}
  \sum_{x} r(x) = 1
\end{eqnarray}
と
\begin{eqnarray}
  \label{10.50}
  r(x) > 0 \ \ \mbox{for all} \ \ x \in \mathcal{X}
\end{eqnarray}
となる.
ここで,ラグランジュ乗数法を用いて,式(\ref{10.50})の正則制約を一時的に無視して最適を求める.
\begin{eqnarray}
  \label{10.51}
  J = \sum_{x} \sum_{y} r(x) p(y|x) \log \frac{q(x|y)}{r(x)} - \lambda \sum_{x} r(x)
\end{eqnarray}
とする.
便宜上,対数は自然対数だると仮定する.
$r(x)$に関して微分すると
\begin{eqnarray}
  \label{10.52}
  \frac{\partial J }{\partial r(x)} = \sum_{x} p(y|x) \log q(x|y) - \log r(x) - 1 - \lambda 
\end{eqnarray}
$\frac{\partial J }{\partial r(x)} = 0$とすると,
\begin{eqnarray}
  \label{10.53}
  \log r(x) = \sum_{x} p(y|x) \log q(x|y) - 1 - \lambda
\end{eqnarray}
または,
\begin{eqnarray}
  \label{10.54}
  r(x) = e{-(\lambda + 1)} \prod_{y} q(x|y) ^{p(y|x)}
\end{eqnarray}
となる.
式(\ref{10.49})の正規化制約を考慮すれば,$\lambda$を排除して
\begin{eqnarray}
  \label{10.55}
  r(x) = \frac{\prod_{y} q(x|y)^{p(y|x)}}{\sum_{x'} \prod_{y} q(x'|y)^{p(y|x')}}
\end{eqnarray}
を得ることができる.
上記の積は,$p(y|x)>0$,$q(x|y)>0$となるすべての$y$についてである.
このことは,上記右辺の分子と分母がともに正であることを意味し,したがって$r(x)>0$となる.
つまり,ラグランジュ乗数を設定する際に無視した式{\ref{10.50}}の正値性制約を,こうして得られた$\mathbf{r}$がたまたま満たすことになる.
10.3.2節(仮)で$f$が凹型であることを示すことにする.
このとき,式(\ref{10.55})で与えられた$\mathbf{r}$は一意であり,$\mathbf{r}$が$A_1$の内部にあるため,与えれた$\mathbf{q} \in A_2$について$f$の最大値を実際に達成する.
式(\ref{10.55})を考慮して,$k \geq 1$に対する$\mathbf{r}^{(k)}$を定義する.
\begin{eqnarray}
  \label{10.56}
  r^{(k)}(x) = \frac{\prod_{y} q^{(k-1)}(x|y)^{p(y|x)}}{\sum_{x'} \prod_{y} q^{(k-1)}(x'|y)^{p(y|x')}}
\end{eqnarray}
ベクトル$\mathbf{r}^{(k)}$と$\mathbf{q}^{(k)}$は,$\mathbf{r}^{(0)}, \mathbf{q}^{(0)}, \mathbf{r}^{(1)}, \mathbf{q}^{(1)}, \mathbf{r}^{(2)}, \mathbf{q}^{(2)}, \cdots$の順に定義され,$A_1$において$\mathbf{r}^{(0)}$が任意に選ばれることを覗いて,配列中の各ベクトルは前のベクトルの関数となる.
$k \geq 1$では$\mathbf{r}^{(k)} \in A_1$,$k \geq 0$では$\mathbf{q}^{(k)} \in A_2$であることを帰納法で示すことになる.
$\mathbf{r}^{(k)} \in A_1$,すなわち$\mathbf{r}^{(k)} > 0$であれば,式(\ref{10.48})から$q^{(k)}(x|y)=0$,すなわち,$\mathbf{q}^{(k)} \in A_2$である場合にのみ$p(x|y)=0$であることがわかる.
一方,$\mathbf{q}^{(k)} \in A_2$ならば,式(\ref{10.56})から$\mathbf{r}^{(k+1)}>0$,すなわち$\mathbf{q}^{(k+1)} \in A_2$であることがわかる.
したがって,すべての$k \geq 0$に対して$\mathbf{R}^{(k)} \in A_1$,$\mathbf{q}^{(k)} \in A_2$が成立する.
$(\mathbf{r}^{(k)}, \mathbf{q}^{(k)})$が決まれば,すべての$k$について$f^{(k)} = f(\mathbf{r}^{(k)}, \mathbf{q}^{(k)})$を計算することができる.
10.3節(仮)で$f^{(k)} \rightarrow C$が示される予定である.

\end{document}